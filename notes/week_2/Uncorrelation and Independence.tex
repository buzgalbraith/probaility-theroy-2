\documentclass{article}
\usepackage[utf8]{inputenc}
\title{vedio 2: Uncorrelation and Independence}
\author{wbg231 }
\date{December 2022}
\newcommand{\R}{$\mathbb{R}$}
\newcommand{\B}{$\beta$}
\newcommand{\A}{$\alpha$}
\newcommand{\D}{\Delta}

\newcommand{\avector}[2]{(#1_2,\ldots,#1_{#2})}
\newcommand{\makedef}[2]{$\textbf{#1}$:#2 }
\usepackage{tikz,graphicx,hyperref,amsmath,amsfonts,amscd,amssymb,bm,cite,epsfig,epsf,url}

\begin{document}

\maketitle

\section{introduction}
\begin{itemize}
\item \href{https://www.youtube.com/watch?v=t9TNCYYK8ck&list=PLBEf5mJtE6KuZ5NBQMuWIMsiOOrV9ibzm&index=62}{video link}
\item we want to establish the difference between Independence and Uncorrelation 
\section{uncorelated data}
\item if$\rho_{a,b}\geq 0$ and $cov(a,b)$ then a and b are positively correlated. there is a positive linear relationship between the two ie they are proportional
\item when $\rho_{a,b}=0$ there is no linear dependence we just guess that mean and the residual is the original data
\section{independent data}
\item if a and b are independent then $cov(a,b)=E[ab]-E[a]E[b]=E[a]E[b]-E[a]E[b]=0$
\item Independence implies uncorelation but it does not go the other way 
\item the logic is more or less that Independence means A and B have no relationship at all uncorealtion means that a and b have no linear relationship
\section{Gaussian random variables}
\item covariance for uncorrelated Gaussian random variables with zero mean and unit variance 
\item $\Sigma=\begin{pmatrix} 1 & \rho\\ \rho & 1 \end{pmatrix}$
\item recall that the off digaonl of $\Sigma$ is the correlation between a and b so we have $\Sigma=\begin{pmatrix} 1 & \rho\\ \rho & 1 \end{pmatrix}=\begin{pmatrix} 1 & 0\\ 0 & 1 \end{pmatrix}=I$
\item then we have $\Sigma^{-1}=I$
\item so we joint pdf $f_{a,b}(a,b)=\frac{1}{2\pi \sqrt{\Sigma}}e^{\frac{-1}{2}\begin{pmatrix}
    a\\b
\end{pmatrix}^t\Sigma^{-1}\begin{pmatrix}
    a\\b
\end{pmatrix}}=\frac{1}{2\pi}e^{-\frac{a^2}+b^2{2}}=\frac{1}{2\pi}e^{-\frac{a^2}{2}}\frac{1}{2\pi}e^{-\frac{b^2}{2}}=f_{a}(a)f_{b}(b)$
\item so in other words Gaussian random variables that are uncorrelated are independent completely
\item keep in mind that this is not always the case there are situations where uncorrelated random variables will not be independent 
\section{uncorrelated residual }
\item recall in the linear regression problem that a and the residual $a-l(a)$ are uncorrelated 
\item however, the residual and a do not have to be independent

\end{itemize}
\end{document}
